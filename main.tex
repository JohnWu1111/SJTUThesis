% !TeX encoding = UTF-8
% !TeX TXS-program:compile = txs:///latexmk/{-pdf} -xelatex
% !TeX TXS-program:recompile-bibliography = txs:///latexmk/{} -v

% 载入 SJTUThesis 模版
\documentclass[type=doctor]{sjtuthesis}
% 选项
%   type=[doctor|master|bachelor],     % 可选(默认:master),论文类型
%   zihao=[-4|5],                      % 可选(默认:-4),正文字号大小
%   lang=[zh|en|de|ja],                % 可选(默认:zh),论文的主要语言
%   review,                            % 可选(默认:关闭),盲审模式
%   [twoside|oneside],                 % 可选(默认:twoside),双页或单页边距模式
%   [openright|openany],               % 可选(默认:openright),奇数页或任意页开始新章
%   math-style=[ISO|TeX],              % 可选 (默认:ISO),数学符号样式

% 论文基本配置,加载宏包等全局配置
% !TEX root = ./main.tex

\sjtusetup{
  %
  %******************************
  % 注意:
  %   1. 配置里面不要出现空行
  %   2. 不需要的配置信息可以删除
  %******************************
  %
  % 信息录入
  %
  info = {%
    %
    % 标题
    %
    zh / title           = {非平衡量子系统的数值研究},
    en / title           = {Numerical Studies of Non-equilibrium quantum system},
    %
    % 标题页标题
    %   可使用“\\”命令手动控制换行
    %
    % zh / display-title   = {上海交通大学学位论文\\ \LaTeX{} 模板示例文档},
    % en / display-title   = {A Sample Document \\ for \LaTeX-based SJTU Thesis Template},
    %
    % 关键词
    %
    zh / keywords        = {非平衡系统, 数值模拟, },
    en / keywords        = {non-equilibrium system, numerical study, },
    %
    % 姓名
    %
    zh / author          = {吴\quad{}烁杭},
    en / author          = {Shuohang Wu},
    %
    % 指导教师
    %
    zh / supervisor      = {蔡子教授},
    en / supervisor      = {Prof.\ Zi Cai},
    %
    % 副指导教师
    %
    % zh / assoc-supervisor  = {某某教授},
    % en / assoc-supervisor  = {Prof.\ Uom Uom},
    %
    % 学号
    %
    id              = {020072910056},
    %
    % 学位
    %   除交叉学科门类外,各一级学科按所属学科门类申请学位
    %   专业学位类别按专业名称申请学位
    %   本科生不需要填写
    %
    zh / degree          = {理学博士},
    en / degree          = {Doctor of Philosophy},
    %
    % 专业
    %
    zh / major           = {物理学},
    en / major           = {Physics},
    %
    % 所属院系
    %
    zh / department      = {物理与天文学院},
    en / department      = {School of Physics and Astronomy},
    %
    % 答辩日期
    %   使用 ISO 格式 (yyyy-mm-dd);默认为当前时间
    %
    % date                 = {2023-05-18},
    %
    % 标题页显示日期
    %   覆盖对应标题页的日期显示,原样输出
    %
    % zh / display-date    = {2023 年 5 月},
    %
    % 资助基金
    %
    % zh / fund  = {
    %                {国家 973 项目 (No.\ 2025CB000000)},
    %                {国家自然科学基金 (No.\ 81120250000)},
    %              },
    % en / fund  = {
    %                {National Basic Research Program of China (Grant No.\ 2025CB000000)},
    %                {National Natural Science Foundation of China (Grant No.\ 81120250000)},
    %              },
  },
  %
  % 风格设置
  %
  style = {%
    %
    % 关键词首行悬挂
    %
    % keywords-format = hang,
  },
  %
  % 名称设置
  %
  name = {
    % bib             = {References},
    % ack             = {谢\hspace{\ccwd}辞},
    % achv            = {攻读学位期间完成的论文},
  },
}

% 使用 BibLaTeX 处理参考文献
%   biblatex-gb7714-2015 常用选项
%     gbnamefmt=lowercase     姓名大小写由输入信息确定
%     gbpub=false             禁用出版信息缺失处理
\usepackage[backend=biber,style=gb7714-2015]{biblatex}
% 文献表字体
\renewcommand{\bibfont}{\zihao{5}\setbaselineskip{16bp}}
% 文献表条目间的间距
\setlength{\bibitemsep}{3bp plus 1pt}
% 导入参考文献数据库
\addbibresource{refs.bib}

% 脚注格式
\usepackage[perpage,bottom,hang]{footmisc}

% 定义图片文件目录与扩展名
\graphicspath{{figures/}}
\DeclareGraphicsExtensions{.pdf,.eps,.png,.jpg,.jpeg}

% 确定浮动对象的位置,可以使用 [H],强制将浮动对象放到这里(可能效果很差)
% \usepackage{float}

% 固定宽度的表格
% \usepackage{tabularx}

% 使用三线表:toprule,midrule,bottomrule。
\usepackage{booktabs}

% 表格中支持跨行
\usepackage{multirow}

% 表格中数字按小数点对齐
\usepackage{dcolumn}
\newcolumntype{d}[1]{D{.}{.}{#1}}

% 使用长表格
\usepackage{longtable}

% 附带脚注的表格
\usepackage{threeparttable}

% 附带脚注的长表格
\usepackage{threeparttablex}

% 算法环境宏包
\usepackage[ruled,vlined,linesnumbered]{algorithm2e}
% \usepackage{algorithm, algorithmicx, algpseudocode}

% 代码环境宏包
\usepackage{listings}
\lstdefinestyle{lstStyleCode}{%
  aboveskip         = \medskipamount,
  belowskip         = \medskipamount,
  basicstyle        = \ttfamily\zihao{-5}\setbaselineskip{12bp},
  commentstyle      = \slshape\color{black!60},
  stringstyle       = \color{green!40!black!100},
  keywordstyle      = \bfseries\color{blue!50!black},
  extendedchars     = false,
  upquote           = true,
  tabsize           = 2,
  showstringspaces  = false,
  xleftmargin       = 1em,
  xrightmargin      = 1em,
  breaklines        = false,
  framexleftmargin  = 1em,
  framexrightmargin = 1em,
  backgroundcolor   = \color{gray!10},
  columns           = flexible,
  keepspaces        = true,
  texcl             = true,
  mathescape        = true
}
\lstnewenvironment{codeblock}[1][]{%
  \lstset{style=lstStyleCode,#1}}{}

% 直立体数学符号
\providecommand{\dd}{\mathop{}\!\mathrm{d}}
\providecommand{\ee}{\mathrm{e}}
\providecommand{\ii}{\mathrm{i}}
\providecommand{\jj}{\mathrm{j}}

% 国际单位制宏包
\usepackage{siunitx}

% 定理环境宏包
\usepackage{amsthm}
% \usepackage{ntheorem}

% 绘图宏包
\usepackage{tikz}
\usetikzlibrary{arrows.meta, shapes.geometric}

% 数据图表宏包
\usepackage{pgfplots}
\pgfplotsset{compat=newest}

% 一些文档中用到的 logo
\usepackage{hologo}
\providecommand{\XeTeX}{\hologo{XeTeX}}
\providecommand{\BibLaTeX}{\textsc{Bib}\LaTeX}

% 借用 ltxdoc 里面的几个命令方便写文档
\DeclareRobustCommand\cs[1]{\texttt{\char`\\#1}}
\providecommand\pkg[1]{{\sffamily#1}}

% hyperref 宏包在最后调用
\usepackage{hyperref}

% E-mail
\providecommand{\email}[1]{\href{mailto:#1}{\urlstyle{tt}\nolinkurl{#1}}}


\begin{document}

%TC:ignore

% 标题页
\maketitle

% 原创性声明及使用授权书
\copyrightpage
% 插入外置原创性声明及使用授权书
% 此时必须在导言区使用 \usepackage{pdfpages}
% \copyrightpage[scans/sample-copyright.pdf]

% 前置部分
\frontmatter

% 摘要
% !TEX root = ../main.tex

\begin{abstract*}[zh]
  中文摘要应该将学位论文的内容要点简短明了地表达出来,应该包含论文中的基本信息,
  体现科研工作的核心思想。摘要内容应涉及本项科研工作的目的和意义、研究方法、研究
  成果、结论及意义。注意突出学位论文中具有创新性的成果和新见解的部分。摘要中不宜
  使用公式、化学结构式、图表和非公知公用的符号和术语,不标注引用文献编号。硕士学
  位论文中文摘要字数为 500 字左右,博士学位论文中文摘要字数为 800 字左右。英文摘
  要内容应与中文摘要内容一致。

  摘要页的下方注明本文的关键词(4 \textasciitilde{} 6 个)。
\end{abstract*}

\begin{abstract*}[en]
  Shanghai Jiao Tong University (SJTU) is a key university in China. SJTU was
  founded in 1896. It is one of the oldest universities in China. The University
  has nurtured large numbers of outstanding figures include JIANG Zemin, DING
  Guangen, QIAN Xuesen, Wu Wenjun, WANG An, etc.

  SJTU has beautiful campuses, Bao Zhaolong Library, Various laboratories. It
  has been actively involved in international academic exchange programs. It is
  the center of CERNet in east China region, through computer networks, SJTU has
  faster and closer connection with the world.
\end{abstract*}


% 目录
\tableofcontents*
% 插图索引
% \listoffigures*
% 表格索引
% \listoftables*
% 算法索引
% \listofalgorithms*

% 符号对照表
% % !TEX root = ../main.tex

\begin{nomenclature*}
\label{chap:symb}

\begin{longtable}{rl}
  $\epsilon$  & 介电常数  \\  
  $\mu$       & 磁导率    \\
  $\epsilon$  & 介电常数  \\
  $\mu$       & 磁导率    \\
  $\epsilon$  & 介电常数  \\
  $\mu$       & 磁导率    \\
  $\epsilon$  & 介电常数  \\
  $\mu$       & 磁导率    \\
  $\epsilon$  & 介电常数  \\
  $\mu$       & 磁导率    \\
  $\epsilon$  & 介电常数  \\
  $\mu$       & 磁导率    \\
  $\epsilon$  & 介电常数  \\
  $\mu$       & 磁导率    \\
  $\epsilon$  & 介电常数  \\
  $\mu$       & 磁导率    \\
  $\epsilon$  & 介电常数  \\
  $\mu$       & 磁导率    \\
  $\epsilon$  & 介电常数  \\
  $\mu$       & 磁导率    \\
  $\epsilon$  & 介电常数  \\
  $\mu$       & 磁导率    \\
  $\epsilon$  & 介电常数  \\
  $\mu$       & 磁导率    \\
  $\epsilon$  & 介电常数  \\
  $\mu$       & 磁导率    \\
  $\epsilon$  & 介电常数  \\
  $\mu$       & 磁导率    \\
  $\epsilon$  & 介电常数  \\
  $\mu$       & 磁导率    \\
  $\epsilon$  & 介电常数  \\
  $\mu$       & 磁导率    \\
  $\epsilon$  & 介电常数  \\
  $\mu$       & 磁导率    \\
  $\epsilon$  & 介电常数  \\
  $\mu$       & 磁导率    \\
  $\epsilon$  & 介电常数  \\
  $\mu$       & 磁导率    \\
  $\epsilon$  & 介电常数  \\
  $\mu$       & 磁导率    \\
  $\epsilon$  & 介电常数  \\
  $\mu$       & 磁导率    \\
  $\epsilon$  & 介电常数  \\
  $\mu$       & 磁导率    \\
  $\epsilon$  & 介电常数  \\
  $\mu$       & 磁导率    \\
  $\epsilon$  & 介电常数  \\
  $\mu$       & 磁导率    \\
  $\epsilon$  & 介电常数  \\
  $\mu$       & 磁导率    \\
  $\epsilon$  & 介电常数  \\
  $\mu$       & 磁导率    \\
  $\epsilon$  & 介电常数  \\
  $\mu$       & 磁导率    \\
\end{longtable}

\end{nomenclature*}


%TC:endignore

% 主体部分
\mainmatter

% 正文内容
% !TEX root = ../main.tex

\chapter{绪论}

(非平衡系统)

\section{非平衡物理}

	\subsection{量子淬火}

\section{自旋玻璃}

\section{局域化现象}

\section{本征态热化假说}

	\subsection{微正则系综与本征态热化假说}

	\subsection{能级差统计分析}

\section{人工量子系统}

\section{能带与拓扑}

	\subsection{量子几何张量与量子度规}


% % !TEX root = ../main.tex

\chapter{数学与引用文献的标注}

\section{数学}

\subsection{数字和单位}

宏包 \pkg{siunitx} 提供了更好的数字和单位支持:
\begin{itemize}
  \item \num{12345.67890}
  \item \complexnum{1+-2i}
  \item \num{.3e45}
  \item \numproduct{1.654 x 2.34 x 3.430}
  \item \unit{kg.m.s^{-1}}
  \item \unit{\micro\meter} $\unit{\micro\meter}$
  \item \unit{\ohm} $\unit{\ohm}$
  \item \numlist{10;20}
  \item \numlist{10;20;30}
  \item \qtylist{0.13;0.67;0.80}{\milli\metre}
  \item \numrange{10}{20}
  \item \qtyrange{10}{20}{\degreeCelsius}
\end{itemize}

\subsection{数学符号和公式}

按照国标 GB/T 3102.11—1993《物理科学和技术中使用的数学符号》,
微分符号 $\dd$ 应使用直立体。除此之外,数学常数也应使用直立体:
\begin{itemize}
  \item 微分符号 $\dd$:\cs{dd}
  \item 圆周率 $\uppi$:\cs{uppi}
  \item 自然对数的底 $\ee$:\cs{ee}
  \item 虚数单位 $\ii$, $\jj$:\cs{ii} \cs{jj}
\end{itemize}

公式应另起一行居中排版。公式后应注明编号,按章顺序编排,编号右端对齐。
\begin{gather}
  \ee^{\ii\uppi} + 1 = 0, \\
  \frac{\dd^2 u}{\dd t^2} = \int f(x) \dd x.
\end{gather}

公式末尾是需要添加标点符号的,至于用逗号还是句号,取决于公式下面一句是接着公式说的,还是另起一句。
\begin{equation}
  \frac{2h}{\uppi}\int_{0}^{\infty}\frac{\sin\left( \omega\delta \right)}{\omega}
  \cos\left( \omega x \right) \dd\omega = 
  \begin{cases}
    h, & \left| x \right| < \delta, \\
    \frac{h}{2}, & x = \pm \delta, \\
    0, & \left| x \right| > \delta.
  \end{cases}
\end{equation}
公式较长时最好在等号“$=$”处转行。
\begin{align}
    & I (X_3; X_4) - I (X_3; X_4 \mid X_1) - I (X_3; X_4 \mid X_2) \nonumber \\
  = & [I (X_3; X_4) - I (X_3; X_4 \mid X_1)] - I (X_3; X_4 \mid \tilde{X}_2) \\
  = & I (X_1; X_3; X_4) - I (X_3; X_4 \mid \tilde{X}_2).
\end{align}

如果在等号处转行难以实现,也可在 $+$、$-$、$\times$、$\div$ 运算符号处转行,转行
时运算符号仅书写于转行式前,不重复书写。
\begin{multline}
  \frac{1}{2} \Delta (f_{ij} f^{ij}) =
    2 \left(\sum_{i<j} \chi_{ij}(\sigma_{i} - \sigma_{j})^{2}
    + f^{ij} \nabla_{j} \nabla_{i} (\Delta f) \right. \\
  \left. + \nabla_{k} f_{ij} \nabla^{k} f^{ij} +
    f^{ij} f^{k} \left[2\nabla_{i}R_{jk}
    - \nabla_{k} R_{ij} \right] \vphantom{\sum_{i<j}} \right).
\end{multline}

\subsection{定理环境}

示例文件中使用 \pkg{amsthm} 宏包配置了定理、引理和证明等环境。用户也可以使用
\pkg{ntheorem} 宏包。

这里举一个“定理”和“证明”的例子。
\begin{theorem}[留数定理]
\label{thm:res}
  假设 $U$ 是复平面上的一个单连通开子集,$a_1, \ldots, a_n$ 是复平面上有限个点,
  $f$ 是定义在 $U \backslash \{a_1, \ldots, a_n\}$ 上的全纯函数,如果 $\gamma$
  是一条把 $a_1, \ldots, a_n$ 包围起来的可求长曲线,但不经过任何一个 $a_k$,并且
  其起点与终点重合,那么:
  \begin{equation}
    \label{eq:res}
    \oint\limits_\gamma f(z)\, \dd z = 2\uppi \ii \sum_{k=1}^n \operatorname{I}(\gamma, a_k) \operatorname{Res}(f, a_k).
  \end{equation}

  如果 $\gamma$ 是若尔当曲线,那么 $\operatorname{I}(\gamma, a_k) = 1$,因此:
  \begin{equation}
    \label{eq:resthm}
    \oint\limits_\gamma f(z)\, \dd z = 2\uppi \ii \sum_{k=1}^n \operatorname{Res}(f, a_k).
  \end{equation}

  在这里,$\operatorname{Res}(f, a_k)$ 表示 $f$ 在点 $a_k$ 的留数,
  $\operatorname{I}(\gamma, a_k)$ 表示 $\gamma$ 关于点 $a_k$ 的卷绕数。卷绕数是
  一个整数,它描述了曲线 $\gamma$ 绕过点 $a_k$ 的次数。如果 $\gamma$ 依逆时针方
  向绕着 $a_k$ 移动,卷绕数就是一个正数,如果 $\gamma$ 根本不绕过 $a_k$,卷绕数
  就是零。

  定理~\ref{thm:res} 的证明。

  \begin{proof}
    首先,由……

    其次,……

    所以……
  \end{proof}
\end{theorem}

\section{引用文献的标注}

按照教务处的要求,参考文献外观应符合国标 GB/T 7714 的要求。模版使用 \BibLaTeX{}
配合 \pkg{biblatex-gb7714-2015} 样式包%
\footnote{\url{https://www.ctan.org/pkg/biblatex-gb7714-2015}}%
控制参考文献的输出样式,后端采用 \pkg{biber} 管理文献。

请注意 \pkg{biblatex-gb7714-2015} 宏包 2016 年 9 月才加入 CTAN,如果你使用的
\TeX{} 系统版本较旧,可能没有包含 \pkg{biblatex-gb7714-2015} 宏包,需要手动安装。
\BibLaTeX{} 与 \pkg{biblatex-gb7714-2015} 目前在活跃地更新,为避免一些兼容性问
题,推荐使用较新的版本。

正文中引用参考文献时,使用 \verb|\cite{key1,key2,key3...}| 可以产生“上标引用的参
考文献”,如 \cite{Yu2001,Cheng1999,LSC1957}。使用
\verb|\parencite{key1,key2,key3...}| 则可以产生水平引用的参考文献,例如
\parencite{Li1999,Jiang1989,Hopkinson1999}。请看下面的例子,将会穿插使用水平的和
上标的参考文献:普通图书\parencite{Yu2001,Jiang1998},论文集、会议录
\cite{CSTAM1990},科技报告\parencite{WHO1970},学位论文\cite{Zhang1998},专利文
献\parencite{Jiang1989,HBLZ2001},专著中析出的文献\cite{Cheng1999,GBT2659},期刊
中析出的文献\parencite{Li1999,Li2000},报纸中析出的文献\cite{Ding2000}, 电子文献
\parencite{Jiang1999,Christine1998,Xiao2001}。

可以使用 \verb|\nocite{key1,key2,key3...}| 将参考文献条目加入到文献表中但不在正
文中引用。使用 \verb|\nocite{*}| 可以将参考文献数据库中的所有条目加入到文献表
中。
\nocite{Yang1999,Schinstock2000,Wen1990,GBT16159}

% % !TeX root = ../main.tex

\chapter{浮动体}

\section{插图}

插图功能是利用 \TeX{} 的特定编译程序提供的机制实现的,不同的编译程序支持不同的图
形方式。有的同学可能听说“\LaTeX{} 只支持 EPS”,事实上这种说法是不准确的。\XeTeX{}
可以很方便地插入 EPS、PDF、PNG、JPEG 格式的图片。

一般图形都是处在浮动环境中。之所以称为浮动是指最终排版效果图形的位置不一定与源文
件中的位置对应,这也是刚使用 \LaTeX{} 同学可能遇到的问题。如果要强制固定浮动图形
的位置,请使用 \pkg{float} 宏包,它提供了 \texttt{[H]} 参数。

\subsection{单个图形}

图要有图题,研究生图题采用中英文对照,并置于图的编号之后,图的编号和图题应置于图
下方的居中位置。引用图应在图题右上角标出文献来源。文中必须有关于本插图的提示,如
“见图~\ref{fig:energy-distrib}”、“如图~\ref{fig:energy-distrib} 所示”等。该页空
白不够排写该图整体时,则可将其后文字部分提前排写,将图移到次页。

\begin{figure}[!htp]
  \centering
  \begin{tikzpicture}
    \begin{axis}[
      width=12cm,
      height=9cm,
      xmin=0, xmax=7,
      xlabel={$r$ (\unit{\milli\metre})},
      ymin=-1000, ymax=11000,
      ylabel={Energy (\unit[per-mode=symbol]{\watt\per\cubic\metre})},
      scaled ticks=false,
      tick label style={
        /pgf/number format/1000 sep=,
        font={\zihao{-5}},
      },
      minor tick num=1,
      tick pos=left,
      tick align=outside,
      tick style={thin,black},
    ]
      \addplot [only marks,mark=square*] 
        table [x={radial}, y={energy}, col sep=comma] 
        {./assets/energy-distrib.csv};
      \node at (2,6000) 
        {$q_{v}=\dfrac{\sigma\omega^{2}|\mathbf{A}|^{2}}{2}$};
    \end{axis}
  \end{tikzpicture}
  \bicaption{内热源沿径向的分布}{Energy distribution along radial}
  \label{fig:energy-distrib}
\end{figure}

\subsection{多个图形}

简单插入多个图形的例子如图~\ref{fig:SRR} 所示。这两个水平并列放置的子图共用一个
图形计数器,没有各自的子图题。

\begin{figure}[!htp]
  \centering
  \includegraphics[height=2cm]{sjtu-vi-badge-red.pdf}
  \hspace{1cm}
  \includegraphics[height=2cm]{sjtu-vi-badge-red.pdf}
  \bicaption{中文题图}{English caption}
  \label{fig:SRR}
\end{figure}

如果多个图形相互独立,并不共用一个图形计数器,那么用 \texttt{minipage} 或者
\texttt{parbox} 就可以,如图~\ref{fig:parallel1} 与图~\ref{fig:parallel2}。

\begin{figure}[!htp]
  \centering
  \begin{minipage}{0.48\textwidth}
    \centering
    \includegraphics[height=1.7cm]{sjtu-vi-name-red.pdf}
    \caption{并排第一个图}
    \label{fig:parallel1}
  \end{minipage}\hfill
  \begin{minipage}{0.48\textwidth}
    \centering
    \includegraphics[height=1.7cm]{sjtu-vi-name-red.pdf}
    \caption{并排第二个图}
    \label{fig:parallel2}
  \end{minipage}
\end{figure}

如果要为共用一个计数器的多个子图添加子图题,建议使用较新的 \pkg{subcaption} 宏
包,不建议使用 \pkg{subfigure} 或 \pkg{subfig} 等宏包。

推荐使用 \pkg{subcaption} 宏包的 \cs{subcaptionbox} 并排子图,子图题置于子图之
下,子图号用 a)、b) 等表示。也可以使用 \pkg{subcaption} 宏包的 \cs{subcaption}
(放在 minipage中,用法同 \cs{caption})。

\pkg{subcaption} 宏包也提供了 \pkg{subfigure} 和 \pkg{subtable} 环境,如
图~\ref{fig:subfigure}。

\begin{figure}[!htp]
  \centering
  \begin{subfigure}{0.3\textwidth}
    \centering
    \includegraphics[height=2cm]{sjtu-vi-badge-red.pdf}
    \caption{校徽}
  \end{subfigure}
  \hspace{1cm}
  \begin{subfigure}{0.4\textwidth}
    \centering
    \includegraphics[height=1.7cm]{sjtu-vi-name-red.pdf}
    \caption{校名。注意这个图略矮些,subfigure 中同一行的子图在顶端对齐。}
  \end{subfigure}
  \caption{包含子图题的范例(使用 subfigure)}
  \label{fig:subfigure}
\end{figure}

搭配 \pkg{bicaption} 宏包时,可以启用 \cs{subcaptionbox} 和 \cs{subcaption} 的双
语变种 \cs{bisubcaptionbox} 和 \cs{bisubcaption},如图~\ref{fig:bisubcaptionbox}
所示。

\begin{figure}[!hbtp]
  \centering
  \bisubcaptionbox{$R_3 = 1.5\text{mm}$ 时轴承的压力分布云图}%
                  {Pressure contour of bearing when $R_3 = 1.5\text{mm}$}%
                  [6.4cm]{\includegraphics[height=3cm]{example-image-a.pdf}}
  \hspace{1cm}
  \bisubcaptionbox{$R_3 = 2.5\text{mm}$ 时轴承的压力分布云图}%
                  {Pressure contour of bearing when $R_3 = 2.5\text{mm}$}%
                  [6.4cm]{\includegraphics[height=3cm]{example-image-b.pdf}}
  \bicaption{包含子图题的范例(使用 subcaptionbox)}
            {Example with subcaptionbox}
  \label{fig:bisubcaptionbox}
\end{figure}


\section{表格}

\subsection{基本表格}

编排表格应简单明了,表达一致,明晰易懂,表文呼应、内容一致。表题置于表上,研究生
学位论文可以用中、英文两种文字居中排写,中文在上,也可以只用中文。

表格的编排建议采用国际通行的三线表\footnote{三线表,以其形式简洁、功能分明、阅读
方便而在科技论文中被推荐使用。三线表通常只有 3 条线,即顶线、底线和栏目线,没有
竖线。}。三线表可以使用 \pkg{booktabs} 提供的 \cs{toprule}、\cs{midrule} 和
\cs{bottomrule}。它们与 \pkg{longtable} 能很好的配合使用。

\begin{table}[!hpt]
  \caption[一个颇为标准的三线表]{一个颇为标准的三线表\footnotemark}
  \label{tab:firstone}
  \centering
  \begin{tabular}{@{}llr@{}} \toprule
    \multicolumn{2}{c}{Item} \\ \cmidrule(r){1-2}
    Animal & Description & Price (\$)\\ \midrule
    Gnat  & per gram  & 13.65 \\
          & each      & 0.01 \\
    Gnu   & stuffed   & 92.50 \\
    Emu   & stuffed   & 33.33 \\
    Armadillo & frozen & 8.99 \\ \bottomrule
  \end{tabular}
\end{table}
\footnotetext{这个例子来自
  \href{https://mirrors.sjtug.sjtu.edu.cn/ctan/macros/latex/contrib/booktabs/booktabs.pdf}%
  {《Publication quality tables in LaTeX》}(\pkg{booktabs} 宏包的文档)。这也是
  一个在表格中使用脚注的例子,请留意与 \pkg{threeparttable} 实现的效果有何不
  同。}

\subsection{复杂表格}

我们经常会在表格下方标注数据来源,或者对表格里面的条目进行解释。可以用
\pkg{threeparttable} 实现带有脚注的表格,如表~\ref{tab:footnote}。

\begin{table}[!htpb]
  \bicaption{一个带有脚注的表格的例子}{A Table with footnotes}
  \label{tab:footnote}
  \centering
  \begin{threeparttable}[b]
     \begin{tabular}{ccd{4}cccc}
      \toprule
      \multirow{2}*{total} & \multicolumn{2}{c}{20\tnote{a}} & \multicolumn{2}{c}{40} & \multicolumn{2}{c}{60} \\
      \cmidrule(lr){2-3}\cmidrule(lr){4-5}\cmidrule(lr){6-7}
      & www & \multicolumn{1}{c}{k} & www & k & www & k \\ % 使用说明符 d 的列会自动进入数学模式,使用 \multicolumn 对文字表头做特殊处理
      \midrule
      & $\underset{(2.12)}{4.22}$ & 120.0140\tnote{b} & 333.15 & 0.0411 & 444.99 & 0.1387 \\
      & 168.6123 & 10.86 & 255.37 & 0.0353 & 376.14 & 0.1058 \\
      & 6.761    & 0.007 & 235.37 & 0.0267 & 348.66 & 0.1010 \\
      \bottomrule
    \end{tabular}
    \begin{tablenotes}
    \item [a] the first note.
    \item [b] the second note.
    \end{tablenotes}
  \end{threeparttable}
\end{table}

如某个表需要转页接排,可以用 \pkg{longtable} 实现。接排时表题省略,表头应重复书
写,并在右上方写“续表 xx”,如表~\ref{tab:performance}。

\begin{ThreePartTable}
  \begin{TableNotes}
    \item[a] 一个脚注
    \item[b] 另一个脚注
  \end{TableNotes}
  \begin{longtable}[c]{c*{6}{r}}
    \bicaption{实验数据}{Experimental data}
    \label{tab:performance} \\
    \toprule
    测试程序 & \multicolumn{1}{c}{正常运行} & \multicolumn{1}{c}{同步}
      & \multicolumn{1}{c}{检查点} & \multicolumn{1}{c}{卷回恢复}
      & \multicolumn{1}{c}{进程迁移} & \multicolumn{1}{c}{检查点} \\
    & \multicolumn{1}{c}{时间 (s)} & \multicolumn{1}{c}{时间 (s)}
      & \multicolumn{1}{c}{时间 (s)} & \multicolumn{1}{c}{时间 (s)}
      & \multicolumn{1}{c}{时间 (s)} &  文件(KB)\\
    \midrule
    \endfirsthead
    \multicolumn{7}{l}{\textbf{续表~\thetable}} \\
    % 英语论文:\multicolumn{7}{r}{\textbf{Table~\thetable~(continued)}} \\
    \toprule
    测试程序 & \multicolumn{1}{c}{正常运行} & \multicolumn{1}{c}{同步}
      & \multicolumn{1}{c}{检查点} & \multicolumn{1}{c}{卷回恢复}
      & \multicolumn{1}{c}{进程迁移} & \multicolumn{1}{c}{检查点} \\
    & \multicolumn{1}{c}{时间 (s)} & \multicolumn{1}{c}{时间 (s)}
      & \multicolumn{1}{c}{时间 (s)} & \multicolumn{1}{c}{时间 (s)}
      & \multicolumn{1}{c}{时间 (s)}&  文件(KB)\\
    \midrule
    \endhead
    \hline
    \multicolumn{7}{r}{续下页}
    \endfoot
    \insertTableNotes
    \endlastfoot
    CG.A.2 & 23.05 & 0.002 & 0.116 & 0.035 & 0.589 & 32491 \\
    CG.A.4 & 15.06 & 0.003 & 0.067 & 0.021 & 0.351 & 18211 \\
    CG.A.8 & 13.38 & 0.004 & 0.072 & 0.023 & 0.210 & 9890 \\
    CG.B.2 & 867.45 & 0.002 & 0.864 & 0.232 & 3.256 & 228562 \\
    CG.B.4 & 501.61 & 0.003 & 0.438 & 0.136 & 2.075 & 123862 \\
    CG.B.8 & 384.65 & 0.004 & 0.457 & 0.108 & 1.235 & 63777 \\
    MG.A.2 & 112.27 & 0.002 & 0.846 & 0.237 & 3.930 & 236473 \\
    MG.A.4 & 59.84 & 0.003 & 0.442 & 0.128 & 2.070 & 123875 \\
    MG.A.8 & 31.38 & 0.003 & 0.476 & 0.114 & 1.041 & 60627 \\
    MG.B.2 & 526.28 & 0.002 & 0.821 & 0.238 & 4.176 & 236635 \\
    MG.B.4 & 280.11 & 0.003 & 0.432 & 0.130 & 1.706 & 123793 \\
    MG.B.8 & 148.29 & 0.003 & 0.442 & 0.116 & 0.893 & 60600 \\
    LU.A.2 & 2116.54 & 0.002 & 0.110 & 0.030 & 0.532 & 28754 \\
    LU.A.4 & 1102.50 & 0.002 & 0.069 & 0.017 & 0.255 & 14915 \\
    LU.A.8 & 574.47 & 0.003 & 0.067 & 0.016 & 0.192 & 8655 \\
    LU.B.2 & 9712.87 & 0.002 & 0.357 & 0.104 & 1.734 & 101975 \\
    LU.B.4 & 4757.80 & 0.003 & 0.190 & 0.056 & 0.808 & 53522 \\
    LU.B.8 & 2444.05 & 0.004 & 0.222 & 0.057 & 0.548 & 30134 \\
    EP.A.2 & 123.81 & 0.002 & 0.010 & 0.003 & 0.074 & 1834 \\
    EP.A.4 & 61.92 & 0.003 & 0.011 & 0.004 & 0.073 & 1743 \\
    EP.A.8 & 31.06 & 0.004 & 0.017 & 0.005 & 0.073 & 1661 \\
    EP.B.2 & 495.49 & 0.001 & 0.009 & 0.003 & 0.196 & 2011 \\
    EP.B.4 & 247.69 & 0.002 & 0.012 & 0.004 & 0.122 & 1663 \\
    EP.B.8 & 126.74 & 0.003 & 0.017 & 0.005 & 0.083 & 1656 \\
    SP.A.2 & 123.81 & 0.002 & 0.010 & 0.003 & 0.074 & 1854 \\
    SP.A.4 & 51.92 & 0.003 & 0.011 & 0.004 & 0.073 & 1543 \\
    SP.A.8 & 31.06 & 0.004 & 0.017 & 0.005 & 0.073 & 1671 \\
    SP.B.2 & 495.49 & 0.001 & 0.009 & 0.003 & 0.196 & 2411 \\
    SP.B.4 \tnote{a} & 247.69 & 0.002 & 0.014 & 0.006 & 0.152 & 2653 \\
    SP.B.8 \tnote{b} & 126.74 & 0.003 & 0.017 & 0.005 & 0.082 & 1755 \\
    \bottomrule
  \end{longtable}
\end{ThreePartTable}

\section{算法环境}

算法环境可以使用 \pkg{algorithms} 宏包或者较新的 \pkg{algorithm2e} 实现。
算法~\ref{algo:algorithm} 是一个使用 \pkg{algorithm2e} 的例子。关于排版算法环境
的具体方法,请阅读相关宏包的官方文档。

\begin{algorithm}[htb]
  \caption{算法示例}
  \label{algo:algorithm}
  \small
  \SetAlgoLined
  \KwData{this text}
  \KwResult{how to write algorithm with \LaTeXe }

  initialization\;
  \While{not at end of this document}{
    read current\;
    \eIf{understand}{
      go to next section\;
      current section becomes this one\;
    }{
      go back to the beginning of current section\;
    }
  }
\end{algorithm}

\section{代码环境}

我们可以在论文中插入算法,但是不建议插入大段的代码。如果确实需要插入代码,建议使
用 \pkg{listings} 宏包。

\begin{codeblock}[language=C]
#include <stdio.h>
#include <unistd.h>
#include <sys/types.h>
#include <sys/wait.h>

int main() {
  pid_t pid;

  switch ((pid = fork())) {
  case -1:
    printf("fork failed\n");
    break;
  case 0:
    /* child calls exec */
    execl("/bin/ls", "ls", "-l", (char*)0);
    printf("execl failed\n");
    break;
  default:
    /* parent uses wait to suspend execution until child finishes */
    wait((int*)0);
    printf("is completed\n");
    break;
  }

  return 0;
}
\end{codeblock}

% % !TEX root = ../main.tex

\chapter{全文总结}

这里是全文总结内容。


%TC:ignore

% 参考文献
\printbibliography[heading=bibintoc]

% 附录
%\appendix

% 附录中图表不加入索引
% \captionsetup{list=no}

% 附录内容
% % !TEX root = ../main.tex

\chapter{Maxwell Equations}

选择二维情况,有如下的偏振矢量:
\begin{subequations}
  \begin{align}
    {\bf E} &= E_z(r, \theta) \hat{\bf z}, \\
    {\bf H} &= H_r(r, \theta) \hat{\bf r} + H_\theta(r, \theta) \hat{\bm\theta}.
  \end{align}
\end{subequations}
对上式求旋度:
\begin{subequations}
  \begin{align}
    \nabla \times {\bf E} &= \frac{1}{r} \frac{\partial E_z}{\partial\theta}
      \hat{\bf r} - \frac{\partial E_z}{\partial r} \hat{\bm\theta}, \\
    \nabla \times {\bf H} &= \left[\frac{1}{r} \frac{\partial}{\partial r}
      (r H_\theta) - \frac{1}{r} \frac{\partial H_r}{\partial\theta} \right]
      \hat{\bf z}.
  \end{align}
\end{subequations}
因为在柱坐标系下,$\overline{\overline\mu}$ 是对角的,所以 Maxwell 方程组中电场
$\bf E$ 的旋度:
\begin{subequations}
  \begin{align}
    & \nabla \times {\bf E} = \ii \omega {\bf B}, \\
    & \frac{1}{r} \frac{\partial E_z}{\partial\theta} \hat{\bf r} -
      \frac{\partial E_z}{\partial r}\hat{\bm\theta} = \ii \omega \mu_r H_r
      \hat{\bf r} + \ii \omega \mu_\theta H_\theta \hat{\bm\theta}.
  \end{align}
\end{subequations}
所以 $\bf H$ 的各个分量可以写为:
\begin{subequations}
  \begin{align}
    H_r &= \frac{1}{\ii \omega \mu_r} \frac{1}{r}
      \frac{\partial E_z}{\partial\theta}, \\
    H_\theta &= -\frac{1}{\ii \omega \mu_\theta}
      \frac{\partial E_z}{\partial r}.
  \end{align}
\end{subequations}
同样地,在柱坐标系下,$\overline{\overline\epsilon}$ 是对角的,所以 Maxwell 方程
组中磁场 $\bf H$ 的旋度:
\begin{subequations}
  \begin{align}
    & \nabla \times {\bf H} = -\ii \omega {\bf D}, \\
    & \left[\frac{1}{r} \frac{\partial}{\partial r}(r H_\theta) - \frac{1}{r}
      \frac{\partial H_r}{\partial\theta} \right] \hat{\bf z} = -\ii \omega
      {\overline{\overline\epsilon}} {\bf E} = -\ii \omega \epsilon_z E_z
      \hat{\bf z}, \\
    & \frac{1}{r} \frac{\partial}{\partial r}(r H_\theta) - \frac{1}{r}
      \frac{\partial H_r}{\partial\theta} = -\ii \omega \epsilon_z E_z.
  \end{align}
\end{subequations}
由此我们可以得到关于 $E_z$ 的波函数方程:
\begin{equation}
  \frac{1}{\mu_\theta \epsilon_z} \frac{1}{r} \frac{\partial}{\partial r}
  \left(r \frac{\partial E_z}{\partial r} \right) + \frac{1}{\mu_r \epsilon_z}
  \frac{1}{r^2} \frac{\partial^2E_z}{\partial\theta^2} +\omega^2 E_z = 0.
\end{equation}

% % !TEX root = ../main.tex

\chapter{绘制流程图}

图~\ref{fig:flow_chart} 是一张流程图示意。使用 \pkg{tikz} 环境,搭配四种预定义节
点(\verb|startstop|、\verb|process|、\verb|decision| 和 \verb|io|),可以容易地
绘制出流程图。

\begin{figure}[!htp]
  \centering
  \input{figures/flow_chart.tex}
  \bicaption{绘制流程图效果}{Flow chart}
  \label{fig:flow_chart}
\end{figure}


% 结尾部分
\backmatter

% 用于盲审的论文需隐去致谢、发表论文、科研成果、简历

% 致谢
% !TEX root = ../main.tex

\begin{acknowledgements}
  感谢那位最先制作出博士学位论文 \LaTeX{} 模板的物理系同学!

  感谢 William Wang 同学对模板移植做出的贡献!

  感谢 \href{https://github.com/weijianwen}{@weijianwen} 学长开创性的工作!

  感谢 \href{https://github.com/sjtug}{@sjtug} 对 0.10 及之后版本的开发和维护工作!

  感谢所有为模板贡献过代码的\href{https://github.com/sjtug/SJTUThesis/graphs/contributors}{同学们}, 以及所有测试和使用模板的各位同学!

  感谢 \LaTeX 和 \href{https://github.com/sjtug/SJTUThesis}{SJTUThesis},帮我节省了不少时间。
\end{acknowledgements}


% 发表论文及科研成果
% 盲审论文中,发表论文及科研成果等仅以第几作者注明即可,不要出现作者或他人姓名
% !TEX root = ../main.tex

\begin{achievements}

	\subsection*{学术论文}

		\begin{bibliolist}{00}
  			\item Chen H, Chan C~T. Acoustic cloaking in three dimensions using acoustic metamaterials[J]. Applied Physics Letters, 2007, 91:183518.
  			\item Chen H, Wu B~I, Zhang B, et al. Electromagnetic Wave Interactions with a Metamaterial Cloak[J]. Physical Review Letters, 2007, 99(6):63903.
		\end{bibliolist}

		\begin{bibliolist*}{00}
  			\item 第一作者. 中文核心期刊论文, 2007.
  			\item 第一作者. EI 国际会议论文, 2006.
		\end{bibliolist*}

\end{achievements}


% 简历
% % !TEX root = ../main.tex

\begin{resume}
  \subsection*{基本情况}
    某某,yyyy 年 mm 月生于 xxxx。

  \subsection*{教育背景}
  \begin{itemize}
    \item yyyy 年 mm 月至今,上海交通大学,博士研究生,xx 专业
    \item yyyy 年 mm 月至 yyyy 年 mm 月,上海交通大学,硕士研究生,xx 专业
    \item yyyy 年 mm 月至 yyyy 年 mm 月,上海交通大学,本科,xx 专业
  \end{itemize}

  \subsection*{研究兴趣}
    \LaTeX{} 排版

  \subsection*{联系方式}
  \begin{itemize}
    \item 地址: 上海市闵行区东川路 800 号,200240
    \item E-mail: \email{john_smith@sjtu.edu.cn}
  \end{itemize}
\end{resume}


% 学士学位论文要求在最后有一个大摘要,单独编页码
% % !TEX root = ../main.tex

\begin{digest}
  An imperial edict issued in 1896 by Emperor Guangxu, established Nanyang
  Public School in Shanghai. The normal school, school of foreign studies,
  middle school and a high school were established. Sheng Xuanhuai, the person
  responsible for proposing the idea to the emperor, became the first president
  and is regarded as the founder of the university.

  During the 1930s, the university gained a reputation of nurturing top
  engineers. After the foundation of People's Republic, some faculties were
  transferred to other universities. A significant amount of its faculty were
  sent in 1956, by the national government, to Xi'an to help build up Xi'an Jiao
  Tong University in western China. Afterwards, the school was officially
  renamed Shanghai Jiao Tong University.

  Since the reform and opening up policy in China, SJTU has taken the lead in
  management reform of institutions for higher education, regaining its vigor
  and vitality with an unprecedented momentum of growth. SJTU includes five
  beautiful campuses, Xuhui, Minhang, Luwan Qibao, and Fahua, taking up an area
  of about \qty{3225833}{\square\metre}. A number of disciplines have been
  advancing towards the top echelon internationally, and a batch of burgeoning
  branches of learning have taken an important position domestically.

  Today SJTU has 31 schools (departments), 63 undergraduate programs, 250
  masters-degree programs, 203 Ph.D. programs, 28 post-doctorate programs, and
  11 state key laboratories and national engineering research centers.

  SJTU boasts a large number of famous scientists and professors, including 35
  academics of the Academy of Sciences and Academy of Engineering, 95 accredited
  professors and chair professors of the ``Cheung Kong Scholars Program'' and
  more than \num{2000} professors and associate professors.

  Its total enrollment of students amounts to \num{35929}, of which \num{1564}
  are international students. There are \num{16802} undergraduates, and
  \num{17563} masters and Ph.D. candidates. After more than a century of
  operation, Jiao Tong University has inherited the old tradition of ``high
  starting points, solid foundation, strict requirements and extensive
  practice.'' Students from SJTU have won top prizes in various competitions,
  including ACM International Collegiate Programming Contest, International
  Mathematical Contest in Modeling and Electronics Design Contests. Famous
  alumni include Jiang Zemin, Lu Dingyi, Ding Guangen, Wang Daohan, Qian Xuesen,
  Wu Wenjun, Zou Taofen, Mao Yisheng, Cai Er, Huang Yanpei, Shao Lizi, Wang An
  and many more. More than 200 of the academics of the Chinese Academy of
  Sciences and Chinese Academy of Engineering are alumni of Jiao Tong
  University.
\end{digest}


%TC:endignore

\end{document}
